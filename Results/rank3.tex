\documentclass[10pt,a4paper]{amsart}

\usepackage{bm}
\usepackage{stmaryrd}

\setlength{\textheight}{650pt}

\begin{document}

\section*{Introduction}

\vspace{-0.05cm}

This paper contains the computation of the motive of the irreducible $\mathrm{SL}_{3}(k)$-character variety of torus knots for any algebraically closed field $k$ of zero characteristic. The calculation is based on the methods introduced in the paper \cite{GPM}. 

The notations used in this paper are the following:
\begin{itemize}
	\item $R_{3}^{\textrm{irr}}$ is the irreducible $\mathrm{SL}_{3}(k)$-representation variety of torus knots, that is, the variety of irreducible representations $\rho: \Gamma \to \mathrm{SL}_{3}(k)$ where $\Gamma = \Gamma_{n,m}$ is the fundamental group of the complement of the $(n,m)$-torus knot (see section {3} of \cite{GPM}).
	\item ${\frak M}_{3}^{\textrm{irr}} = R_{3}^{\textrm{irr}} \sslash \mathrm{SL}_{3}(k)$ is the irreducible $\mathrm{SL}_{3}(k)$-character variety of torus knots, that is, the moduli space of representations (see section {3} of \cite{GPM}).
	\item $\kappa = (\bm{\epsilon}, \bm{\varepsilon})$ is a configuration of eigenvalues, that is a collection of possible eigenvalues for the matrices $A$ and $B$ of a torus knot representation $\rho = (A,B)$ (see section {3} of \cite{GPM}).
	\item $\tau$ is the type of a semi-simple filtration of a torus knot representation (see section 2.1 of \cite{GPM}).
	\item $\xi$ is the shape of the type $\tau$, that is the collection of dimensions and multiplicities of each isotypic component (see section 2.1 of \cite{GPM}).
	\item $\sigma_A$ are the collections of eigenvalues of $A$ for each isotypic component of a torus knot representation $\rho = (A,B)$ (see section 7.1 of \cite{GPM}).
	\item $\sigma_B$ are the collections of eigenvalues of $B$ for each isotypic component of a torus knot representation $\rho = (A,B)$ (see section 7.1 of \cite{GPM}).
	\item $\mathcal{M}_{\tau}$ is the space parametrizing possible completions of a semi-simple representation to a general representation of type $\tau$ (see section 4 of \cite{GPM})
	\item $\mathcal{G}_{\tau}$ is the gauge group acting on $\mathcal{M}_{\tau} \times \mathrm{SL}_{3}(k)$ that identifies isomorphic completions (see section 4 of \cite{GPM}).
	\item ${\frak M}_{\tau}^{\textrm{irr}}$ is the variety of possible semi-simplifications of a representation of type $\tau$ (see section 4 of \cite{GPM}).
	\item $R(\tau)$ is the variety of representations of type $\tau$.
	\item $m_{\kappa}(\tau)$ is the multiplicity of the type $\tau$, that is the number of isomorphic components $R(\tau')$ of types $\tau'$ with the same shape as $\tau$ but whose eigenvalues are given by a permutation of the ones of $\tau$ that preserves their multiplicity (see section 5 of \cite{GPM}).
	\item $C_{\pi, \pi'}$ are the number of isomorphic components given by configurations of eigenvalues with the same structure of repeated eigenvalues (see Section 6 of \cite{GPM}). Here, $\pi, \pi'$ are two partitions of ${3}$ that determine the number of repeated eigenvalues of the matrices $A$ and $B$ of a representation $\rho = (A,B)$. If $\pi = \left\{1^{e_1}, \ldots, {3}^{e_{3}}\right\}$ and $\pi' = \big\{1^{e_1'}, \ldots, {3}^{e_{3}'}\big\}$ we have the following characterization in terms of multinomial numbers (Theorem 6.8 of \cite{GPM})
	$$
		C_{\pi, \pi'} = \frac{{3}}{nm} \begin{pmatrix}
	n\\
	e_1, \ldots, e_{3} \end{pmatrix}\begin{pmatrix}
	m\\
	e_1', \ldots, e_{3}'\end{pmatrix}.
	$$
\end{itemize}

Combinatorial formulas for the motives $[\mathcal{M}_{\tau}],[\mathcal{G}_{\tau}]$ and $[{\frak M}_{\tau}^{\textrm{irr}}]$ are described in section 5 of \cite{GPM} in terms of the structure of the type $\tau$.

The structure of the paper is as follows. Each section describes the count of the motive $[{\frak M}_\kappa]$ for a possible configuration of eigenvalues $\kappa$. For that purpose, we analyze all the types $\tau$ compatible with $\kappa$ and compute the motives $[R(\tau)]$. A configuration of eigenvalues $\kappa$ not appearing as a section of the paper means that $R^{\textrm{irr}}_\kappa = \emptyset$ (see Remark {3}.5 and Proposition 8.1 of \cite{GPM}). In the final section of this paper, we summarize the results for each configuration $\kappa$ and we provide the final result depending on the combinatorial coefficients $C_{\pi, \pi'}$.

\textbf{Warning: }The script generating this paper is only valid for rank $\leq 4$. The result for higher rank may not be correct.

\newpage{}


\section{Configuration $\bm{\epsilon} = (\epsilon_1,\epsilon_1,\epsilon_2)$ and $\bm{\varepsilon} =(\varepsilon_1,\varepsilon_2,\varepsilon_3)$}
\noindent\rule{8cm}{0.4pt}

$$\xi = ({(2, 1), (1, 1)}),\quad \sigma_A = ({{\epsilon_1, \epsilon_2}, {\epsilon_1}}),\quad \sigma_B = ({{\epsilon_1, \epsilon_3}, {\epsilon_2}}).$$

\begin{itemize}
 \item $[\mathcal{M}_{\tau}] = 1 .$

 \item $[\mathcal{G}_{\tau}] = {\left(q - 1\right)}^{2} .$

 \item $[{\frak M}_{\tau}^{\textrm{irr}}] = q - 2 .$

 \item $[R(\tau)] = q^{8} - q^{7} - 2 \, q^{6} - q^{5} + q^{4} + 2 \, q^{3} $

 \item $m_{\kappa}(\tau) = 3 .$

 \end{itemize}
\noindent\rule{8cm}{0.4pt}

$$\xi = ({(1, 1), (1, 1), (1, 1)}),\quad \sigma_A = ({{\epsilon_1}, {\epsilon_1}, {\epsilon_2}}),\quad \sigma_B = ({{\epsilon_1}, {\epsilon_2}, {\epsilon_3}}).$$

\begin{itemize}
 \item $[\mathcal{M}_{\tau}] = 1 .$

 \item $[\mathcal{G}_{\tau}] = {\left(q - 1\right)}^{3} .$

 \item $[{\frak M}_{\tau}^{\textrm{irr}}] = 1 .$

 \item $[R(\tau)] = q^{6} + 2 \, q^{5} + 2 \, q^{4} + q^{3} $

 \item $m_{\kappa}(\tau) = 3 .$

 \end{itemize}
\noindent\rule{8cm}{0.4pt}

$$\xi = ({(1, 1), (1, 1)}, {(1, 1)}),\quad \sigma_A = ({{\epsilon_1}, {\epsilon_1}}, {{\epsilon_2}}),\quad \sigma_B = ({{\epsilon_1}, {\epsilon_2}}, {{\epsilon_3}}).$$

\begin{itemize}
 \item $[\mathcal{M}_{\tau}] = q^{2} - 1 .$

 \item $[\mathcal{G}_{\tau}] = {\left(q - 1\right)}^{3} .$

 \item $[{\frak M}_{\tau}^{\textrm{irr}}] = 1 .$

 \item $[R(\tau)] = q^{8} + 2 \, q^{7} + q^{6} - q^{5} - 2 \, q^{4} - q^{3} $

 \item $m_{\kappa}(\tau) = 3 .$

 \end{itemize}
\noindent\rule{8cm}{0.4pt}

$$\xi = ({(2, 1)}, {(1, 1)}),\quad \sigma_A = ({{\epsilon_1, \epsilon_2}}, {{\epsilon_1}}),\quad \sigma_B = ({{\epsilon_1, \epsilon_3}}, {{\epsilon_2}}).$$

\begin{itemize}
 \item $[\mathcal{M}_{\tau}] = q^{2} - q .$

 \item $[\mathcal{G}_{\tau}] = {\left(q - 1\right)}^{2} q .$

 \item $[{\frak M}_{\tau}^{\textrm{irr}}] = q - 2 .$

 \item $[R(\tau)] = q^{9} - 2 \, q^{8} - q^{7} + q^{6} + 2 \, q^{5} + q^{4} - 2 \, q^{3} $

 \item $m_{\kappa}(\tau) = 3 .$

 \end{itemize}
\noindent\rule{8cm}{0.4pt}

$$\xi = ({(1, 1), (1, 1)}, {(1, 1)}),\quad \sigma_A = ({{\epsilon_1}, {\epsilon_2}}, {{\epsilon_1}}),\quad \sigma_B = ({{\epsilon_1}, {\epsilon_3}}, {{\epsilon_2}}).$$

\begin{itemize}
 \item $[\mathcal{M}_{\tau}] = q^{2} - q .$

 \item $[\mathcal{G}_{\tau}] = {\left(q - 1\right)}^{3} q .$

 \item $[{\frak M}_{\tau}^{\textrm{irr}}] = 1 .$

 \item $[R(\tau)] = q^{7} + q^{6} - q^{4} - q^{3} $

 \item $m_{\kappa}(\tau) = 6 .$

 \end{itemize}
\noindent\rule{8cm}{0.4pt}

$$\xi = ({(1, 1)}, {(2, 1)}),\quad \sigma_A = ({{\epsilon_1}}, {{\epsilon_1, \epsilon_2}}),\quad \sigma_B = ({{\epsilon_1}}, {{\epsilon_2, \epsilon_3}}).$$

\begin{itemize}
 \item $[\mathcal{M}_{\tau}] = q^{2} - q .$

 \item $[\mathcal{G}_{\tau}] = {\left(q - 1\right)}^{2} q .$

 \item $[{\frak M}_{\tau}^{\textrm{irr}}] = q - 2 .$

 \item $[R(\tau)] = q^{9} - 2 \, q^{8} - q^{7} + q^{6} + 2 \, q^{5} + q^{4} - 2 \, q^{3} $

 \item $m_{\kappa}(\tau) = 3 .$

 \end{itemize}
\noindent\rule{8cm}{0.4pt}

$$\xi = ({(1, 1)}, {(1, 1), (1, 1)}),\quad \sigma_A = ({{\epsilon_2}}, {{\epsilon_1}, {\epsilon_1}}),\quad \sigma_B = ({{\epsilon_3}}, {{\epsilon_1}, {\epsilon_2}}).$$

\begin{itemize}
 \item $[\mathcal{M}_{\tau}] = {\left(q - 1\right)}^{2} .$

 \item $[\mathcal{G}_{\tau}] = {\left(q - 1\right)}^{3} .$

 \item $[{\frak M}_{\tau}^{\textrm{irr}}] = 1 .$

 \item $[R(\tau)] = q^{8} - q^{6} - q^{5} + q^{3} $

 \item $m_{\kappa}(\tau) = 3 .$

 \end{itemize}
\noindent\rule{8cm}{0.4pt}

$$\xi = ({(1, 1)}, {(1, 1)}, {(1, 1)}),\quad \sigma_A = ({{\epsilon_1}}, {{\epsilon_2}}, {{\epsilon_1}}),\quad \sigma_B = ({{\epsilon_1}}, {{\epsilon_3}}, {{\epsilon_2}}).$$

\begin{itemize}
 \item $[\mathcal{M}_{\tau}] = {\left(q - 1\right)}^{2} q .$

 \item $[\mathcal{G}_{\tau}] = {\left(q - 1\right)}^{3} q .$

 \item $[{\frak M}_{\tau}^{\textrm{irr}}] = 1 .$

 \item $[R(\tau)] = q^{8} - q^{6} - q^{5} + q^{3} $

 \item $m_{\kappa}(\tau) = 6 .$

 \end{itemize}
\noindent\rule{9cm}{2pt}\vspace{0.2cm}

\noindent\textbf{Total count of $\kappa = ((\epsilon_1,\epsilon_1,\epsilon_2), (\varepsilon_1,\varepsilon_2,\varepsilon_3))$}\medskip

${[R_{\kappa}^{\textrm{red}}]} = 6 \, q^{9} + 3 \, q^{8} + 3 \, q^{7} + 3 \, q^{6} + 3 \, q^{5} + 3 \, q^{4} - 3 \, q^{3} ,$

${[R_{\kappa}^{\textrm{irr}}]} = q^{10} - 3 \, q^{9} + 2 \, q^{8} + 2 \, q^{7} - 2 \, q^{5} - 3 \, q^{4} + 3 \, q^{3} ,$

${[R_{\kappa}]} = q^{10} + 3 \, q^{9} + 5 \, q^{8} + 5 \, q^{7} + 3 \, q^{6} + q^{5} ,$

${[{\frak M}_{\kappa}]} = q^{2} - 3 \, q + 3 .$

\newpage{}

\section{Configuration $\bm{\epsilon} = (\epsilon_1,\epsilon_2,\epsilon_3)$ and $\bm{\varepsilon} =(\varepsilon_1,\varepsilon_1,\varepsilon_2)$}
\noindent\rule{8cm}{0.4pt}

$$\xi = ({(2, 1), (1, 1)}),\quad \sigma_A = ({{\epsilon_1, \epsilon_2}, {\epsilon_3}}),\quad \sigma_B = ({{\epsilon_2, \epsilon_1}, {\epsilon_1}}).$$

\begin{itemize}
 \item $[\mathcal{M}_{\tau}] = 1 .$

 \item $[\mathcal{G}_{\tau}] = {\left(q - 1\right)}^{2} .$

 \item $[{\frak M}_{\tau}^{\textrm{irr}}] = q - 2 .$

 \item $[R(\tau)] = q^{8} - q^{7} - 2 \, q^{6} - q^{5} + q^{4} + 2 \, q^{3} $

 \item $m_{\kappa}(\tau) = 3 .$

 \end{itemize}
\noindent\rule{8cm}{0.4pt}

$$\xi = ({(1, 1), (1, 1), (1, 1)}),\quad \sigma_A = ({{\epsilon_1}, {\epsilon_2}, {\epsilon_3}}),\quad \sigma_B = ({{\epsilon_1}, {\epsilon_1}, {\epsilon_2}}).$$

\begin{itemize}
 \item $[\mathcal{M}_{\tau}] = 1 .$

 \item $[\mathcal{G}_{\tau}] = {\left(q - 1\right)}^{3} .$

 \item $[{\frak M}_{\tau}^{\textrm{irr}}] = 1 .$

 \item $[R(\tau)] = q^{6} + 2 \, q^{5} + 2 \, q^{4} + q^{3} $

 \item $m_{\kappa}(\tau) = 3 .$

 \end{itemize}
\noindent\rule{8cm}{0.4pt}

$$\xi = ({(2, 1)}, {(1, 1)}),\quad \sigma_A = ({{\epsilon_1, \epsilon_2}}, {{\epsilon_3}}),\quad \sigma_B = ({{\epsilon_2, \epsilon_1}}, {{\epsilon_1}}).$$

\begin{itemize}
 \item $[\mathcal{M}_{\tau}] = q - 1 .$

 \item $[\mathcal{G}_{\tau}] = {\left(q - 1\right)}^{2} .$

 \item $[{\frak M}_{\tau}^{\textrm{irr}}] = q - 2 .$

 \item $[R(\tau)] = q^{9} - 2 \, q^{8} - q^{7} + q^{6} + 2 \, q^{5} + q^{4} - 2 \, q^{3} $

 \item $m_{\kappa}(\tau) = 3 .$

 \end{itemize}
\noindent\rule{8cm}{0.4pt}

$$\xi = ({(1, 1), (1, 1)}, {(1, 1)}),\quad \sigma_A = ({{\epsilon_1}, {\epsilon_2}}, {{\epsilon_3}}),\quad \sigma_B = ({{\epsilon_1}, {\epsilon_1}}, {{\epsilon_2}}).$$

\begin{itemize}
 \item $[\mathcal{M}_{\tau}] = q^{2} - 1 .$

 \item $[\mathcal{G}_{\tau}] = {\left(q - 1\right)}^{3} .$

 \item $[{\frak M}_{\tau}^{\textrm{irr}}] = 1 .$

 \item $[R(\tau)] = q^{8} + 2 \, q^{7} + q^{6} - q^{5} - 2 \, q^{4} - q^{3} $

 \item $m_{\kappa}(\tau) = 3 .$

 \end{itemize}
\noindent\rule{8cm}{0.4pt}

$$\xi = ({(1, 1), (1, 1)}, {(1, 1)}),\quad \sigma_A = ({{\epsilon_1}, {\epsilon_2}}, {{\epsilon_3}}),\quad \sigma_B = ({{\epsilon_2}, {\epsilon_1}}, {{\epsilon_1}}).$$

\begin{itemize}
 \item $[\mathcal{M}_{\tau}] = q - 1 .$

 \item $[\mathcal{G}_{\tau}] = {\left(q - 1\right)}^{3} .$

 \item $[{\frak M}_{\tau}^{\textrm{irr}}] = 1 .$

 \item $[R(\tau)] = q^{7} + q^{6} - q^{4} - q^{3} $

 \item $m_{\kappa}(\tau) = 6 .$

 \end{itemize}
\noindent\rule{8cm}{0.4pt}

$$\xi = ({(1, 1)}, {(2, 1)}),\quad \sigma_A = ({{\epsilon_1}}, {{\epsilon_2, \epsilon_3}}),\quad \sigma_B = ({{\epsilon_1}}, {{\epsilon_1, \epsilon_2}}).$$

\begin{itemize}
 \item $[\mathcal{M}_{\tau}] = q - 1 .$

 \item $[\mathcal{G}_{\tau}] = {\left(q - 1\right)}^{2} .$

 \item $[{\frak M}_{\tau}^{\textrm{irr}}] = q - 2 .$

 \item $[R(\tau)] = q^{9} - 2 \, q^{8} - q^{7} + q^{6} + 2 \, q^{5} + q^{4} - 2 \, q^{3} $

 \item $m_{\kappa}(\tau) = 3 .$

 \end{itemize}
\noindent\rule{8cm}{0.4pt}

$$\xi = ({(1, 1)}, {(1, 1), (1, 1)}),\quad \sigma_A = ({{\epsilon_1}}, {{\epsilon_2}, {\epsilon_3}}),\quad \sigma_B = ({{\epsilon_2}}, {{\epsilon_1}, {\epsilon_1}}).$$

\begin{itemize}
 \item $[\mathcal{M}_{\tau}] = {\left(q - 1\right)}^{2} .$

 \item $[\mathcal{G}_{\tau}] = {\left(q - 1\right)}^{3} .$

 \item $[{\frak M}_{\tau}^{\textrm{irr}}] = 1 .$

 \item $[R(\tau)] = q^{8} - q^{6} - q^{5} + q^{3} $

 \item $m_{\kappa}(\tau) = 3 .$

 \end{itemize}
\noindent\rule{8cm}{0.4pt}

$$\xi = ({(1, 1)}, {(1, 1)}, {(1, 1)}),\quad \sigma_A = ({{\epsilon_1}}, {{\epsilon_2}}, {{\epsilon_3}}),\quad \sigma_B = ({{\epsilon_1}}, {{\epsilon_2}}, {{\epsilon_1}}).$$

\begin{itemize}
 \item $[\mathcal{M}_{\tau}] = {\left(q - 1\right)}^{2} .$

 \item $[\mathcal{G}_{\tau}] = {\left(q - 1\right)}^{3} .$

 \item $[{\frak M}_{\tau}^{\textrm{irr}}] = 1 .$

 \item $[R(\tau)] = q^{8} - q^{6} - q^{5} + q^{3} $

 \item $m_{\kappa}(\tau) = 6 .$

 \end{itemize}
\noindent\rule{9cm}{2pt}\vspace{0.2cm}

\noindent\textbf{Total count of $\kappa = ((\epsilon_1,\epsilon_2,\epsilon_3), (\varepsilon_1,\varepsilon_1,\varepsilon_2))$}\medskip

${[R_{\kappa}^{\textrm{red}}]} = 6 \, q^{9} + 3 \, q^{8} + 3 \, q^{7} + 3 \, q^{6} + 3 \, q^{5} + 3 \, q^{4} - 3 \, q^{3} ,$

${[R_{\kappa}^{\textrm{irr}}]} = q^{10} - 3 \, q^{9} + 2 \, q^{8} + 2 \, q^{7} - 2 \, q^{5} - 3 \, q^{4} + 3 \, q^{3} ,$

${[R_{\kappa}]} = q^{10} + 3 \, q^{9} + 5 \, q^{8} + 5 \, q^{7} + 3 \, q^{6} + q^{5} ,$

${[{\frak M}_{\kappa}]} = q^{2} - 3 \, q + 3 .$

\newpage{}

\section{Configuration $\bm{\epsilon} = (\epsilon_1,\epsilon_2,\epsilon_3)$ and $\bm{\varepsilon} =(\varepsilon_1,\varepsilon_2,\varepsilon_3)$}
\noindent\rule{8cm}{0.4pt}

$$\xi = ({(2, 1), (1, 1)}),\quad \sigma_A = ({{\epsilon_1, \epsilon_2}, {\epsilon_3}}),\quad \sigma_B = ({{\epsilon_1, \epsilon_2}, {\epsilon_3}}).$$

\begin{itemize}
 \item $[\mathcal{M}_{\tau}] = 1 .$

 \item $[\mathcal{G}_{\tau}] = {\left(q - 1\right)}^{2} .$

 \item $[{\frak M}_{\tau}^{\textrm{irr}}] = q - 2 .$

 \item $[R(\tau)] = q^{8} - q^{7} - 2 \, q^{6} - q^{5} + q^{4} + 2 \, q^{3} $

 \item $m_{\kappa}(\tau) = 9 .$

 \end{itemize}
\noindent\rule{8cm}{0.4pt}

$$\xi = ({(1, 1), (1, 1), (1, 1)}),\quad \sigma_A = ({{\epsilon_1}, {\epsilon_2}, {\epsilon_3}}),\quad \sigma_B = ({{\epsilon_1}, {\epsilon_2}, {\epsilon_3}}).$$

\begin{itemize}
 \item $[\mathcal{M}_{\tau}] = 1 .$

 \item $[\mathcal{G}_{\tau}] = {\left(q - 1\right)}^{3} .$

 \item $[{\frak M}_{\tau}^{\textrm{irr}}] = 1 .$

 \item $[R(\tau)] = q^{6} + 2 \, q^{5} + 2 \, q^{4} + q^{3} $

 \item $m_{\kappa}(\tau) = 6 .$

 \end{itemize}
\noindent\rule{8cm}{0.4pt}

$$\xi = ({(2, 1)}, {(1, 1)}),\quad \sigma_A = ({{\epsilon_1, \epsilon_2}}, {{\epsilon_3}}),\quad \sigma_B = ({{\epsilon_1, \epsilon_2}}, {{\epsilon_3}}).$$

\begin{itemize}
 \item $[\mathcal{M}_{\tau}] = q^{2} - 1 .$

 \item $[\mathcal{G}_{\tau}] = {\left(q - 1\right)}^{2} .$

 \item $[{\frak M}_{\tau}^{\textrm{irr}}] = q - 2 .$

 \item $[R(\tau)] = q^{10} - q^{9} - 3 \, q^{8} + 3 \, q^{6} + 3 \, q^{5} - q^{4} - 2 \, q^{3} $

 \item $m_{\kappa}(\tau) = 9 .$

 \end{itemize}
\noindent\rule{8cm}{0.4pt}

$$\xi = ({(1, 1), (1, 1)}, {(1, 1)}),\quad \sigma_A = ({{\epsilon_1}, {\epsilon_2}}, {{\epsilon_3}}),\quad \sigma_B = ({{\epsilon_1}, {\epsilon_2}}, {{\epsilon_3}}).$$

\begin{itemize}
 \item $[\mathcal{M}_{\tau}] = q^{2} - 1 .$

 \item $[\mathcal{G}_{\tau}] = {\left(q - 1\right)}^{3} .$

 \item $[{\frak M}_{\tau}^{\textrm{irr}}] = 1 .$

 \item $[R(\tau)] = q^{8} + 2 \, q^{7} + q^{6} - q^{5} - 2 \, q^{4} - q^{3} $

 \item $m_{\kappa}(\tau) = 18 .$

 \end{itemize}
\noindent\rule{8cm}{0.4pt}

$$\xi = ({(1, 1)}, {(2, 1)}),\quad \sigma_A = ({{\epsilon_1}}, {{\epsilon_2, \epsilon_3}}),\quad \sigma_B = ({{\epsilon_1}}, {{\epsilon_2, \epsilon_3}}).$$

\begin{itemize}
 \item $[\mathcal{M}_{\tau}] = q^{2} - 1 .$

 \item $[\mathcal{G}_{\tau}] = {\left(q - 1\right)}^{2} .$

 \item $[{\frak M}_{\tau}^{\textrm{irr}}] = q - 2 .$

 \item $[R(\tau)] = q^{10} - q^{9} - 3 \, q^{8} + 3 \, q^{6} + 3 \, q^{5} - q^{4} - 2 \, q^{3} $

 \item $m_{\kappa}(\tau) = 9 .$

 \end{itemize}
\noindent\rule{8cm}{0.4pt}

$$\xi = ({(1, 1)}, {(1, 1), (1, 1)}),\quad \sigma_A = ({{\epsilon_1}}, {{\epsilon_2}, {\epsilon_3}}),\quad \sigma_B = ({{\epsilon_1}}, {{\epsilon_2}, {\epsilon_3}}).$$

\begin{itemize}
 \item $[\mathcal{M}_{\tau}] = {\left(q - 1\right)}^{2} .$

 \item $[\mathcal{G}_{\tau}] = {\left(q - 1\right)}^{3} .$

 \item $[{\frak M}_{\tau}^{\textrm{irr}}] = 1 .$

 \item $[R(\tau)] = q^{8} - q^{6} - q^{5} + q^{3} $

 \item $m_{\kappa}(\tau) = 18 .$

 \end{itemize}
\noindent\rule{8cm}{0.4pt}

$$\xi = ({(1, 1)}, {(1, 1)}, {(1, 1)}),\quad \sigma_A = ({{\epsilon_1}}, {{\epsilon_2}}, {{\epsilon_3}}),\quad \sigma_B = ({{\epsilon_1}}, {{\epsilon_2}}, {{\epsilon_3}}).$$

\begin{itemize}
 \item $[\mathcal{M}_{\tau}] = {\left(q - 1\right)}^{2} q .$

 \item $[\mathcal{G}_{\tau}] = {\left(q - 1\right)}^{3} .$

 \item $[{\frak M}_{\tau}^{\textrm{irr}}] = 1 .$

 \item $[R(\tau)] = q^{9} - q^{7} - q^{6} + q^{4} $

 \item $m_{\kappa}(\tau) = 36 .$

 \end{itemize}
\noindent\rule{9cm}{2pt}\vspace{0.2cm}

\noindent\textbf{Total count of $\kappa = ((\epsilon_1,\epsilon_2,\epsilon_3), (\varepsilon_1,\varepsilon_2,\varepsilon_3))$}\medskip

${[R_{\kappa}^{\textrm{red}}]} = 18 \, q^{10} + 18 \, q^{9} - 9 \, q^{8} - 9 \, q^{7} + 6 \, q^{6} + 21 \, q^{5} + 3 \, q^{4} - 12 \, q^{3} ,$

${[R_{\kappa}^{\textrm{irr}}]} = q^{12} + 4 \, q^{11} - 10 \, q^{10} - 8 \, q^{9} + 17 \, q^{8} + 13 \, q^{7} - 5 \, q^{6} - 21 \, q^{5} - 3 \, q^{4} + 12 \, q^{3} ,$

${[R_{\kappa}]} = q^{12} + 4 \, q^{11} + 8 \, q^{10} + 10 \, q^{9} + 8 \, q^{8} + 4 \, q^{7} + q^{6} ,$

${[{\frak M}_{\kappa}]} = q^{4} + 4 \, q^{3} - 9 \, q^{2} - 3 \, q + 12 .$

\newpage{}

\section*{Summary}
$[R_{(\epsilon_1,\epsilon_1,\epsilon_2),(\varepsilon_1,\varepsilon_2,\varepsilon_3)}^{\textrm{irr}}] = q^{10} - 3 \, q^{9} + 2 \, q^{8} + 2 \, q^{7} - 2 \, q^{5} - 3 \, q^{4} + 3 \, q^{3} .$

$[R_{(\epsilon_1,\epsilon_2,\epsilon_3),(\varepsilon_1,\varepsilon_1,\varepsilon_2)}^{\textrm{irr}}] = q^{10} - 3 \, q^{9} + 2 \, q^{8} + 2 \, q^{7} - 2 \, q^{5} - 3 \, q^{4} + 3 \, q^{3} .$

$[R_{(\epsilon_1,\epsilon_2,\epsilon_3),(\varepsilon_1,\varepsilon_2,\varepsilon_3)}^{\textrm{irr}}] = q^{12} + 4 \, q^{11} - 10 \, q^{10} - 8 \, q^{9} + 17 \, q^{8} + 13 \, q^{7} - 5 \, q^{6} - 21 \, q^{5} - 3 \, q^{4} + 12 \, q^{3} .$


\medskip\textbf{Final result representations}.\medskip

$[R_{3}^{\textrm{irr}}] = (q^{10} - 3 \, q^{9} + 2 \, q^{8} + 2 \, q^{7} - 2 \, q^{5} - 3 \, q^{4} + 3 \, q^{3}) {C_{(1, 2), (1, 1, 1)}} + (q^{10} - 3 \, q^{9} + 2 \, q^{8} + 2 \, q^{7} - 2 \, q^{5} - 3 \, q^{4} + 3 \, q^{3}) {C_{(1, 1, 1), (1, 2)}} + (q^{12} + 4 \, q^{11} - 10 \, q^{10} - 8 \, q^{9} + 17 \, q^{8} + 13 \, q^{7} - 5 \, q^{6} - 21 \, q^{5} - 3 \, q^{4} + 12 \, q^{3}) {C_{(1, 1, 1), (1, 1, 1)}}.$

\medskip\textbf{Final result characters}.\medskip

$[{\frak M}_{3}^{\textrm{irr}}] = (q^{2} - 3 \, q + 3) {C_{(1, 2), (1, 1, 1)}} + (q^{2} - 3 \, q + 3) {C_{(1, 1, 1), (1, 2)}} + (q^{4} + 4 \, q^{3} - 9 \, q^{2} - 3 \, q + 12) {C_{(1, 1, 1), (1, 1, 1)}}.$

\begin{thebibliography}{11}

\bibitem{GPM}
\'A. Gonz\'alez-Prieto and V. Mu\~noz, \emph{Motive of the $\mathrm{SL}_4(\mathbb{C})$-character variety of torus knots}, arXiv.
\end{thebibliography}

\end{document}
